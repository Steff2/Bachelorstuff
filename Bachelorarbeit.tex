%1. Einführung (Einführung in das Thema (kurz), Ziele der Arbeit und   
%Aufbau der Arbeit)

%2. Grundlagen (soweit, wie man das braucht, also: was ist   
%Brachytherapie, wie geht man da vor, was sind Spieleengines und   
%welche größeren gibt es davon).

%3. Stand der Technik: Was gibt es zum Thema Simulatoren in der   
%Medizin (da kann ich Dich auch mit Material versorgen)


%4. Material und Methoden (wie bist Du vorgegangen:

%a.Spezifikation

%b. Auswahl der Spieleengine (Kriterien finden aus der Spezifikation,  
%  nach denen Du auswählst)
  
%c. Grundaufbau/Architektur: Wie sieht das im Großen und Ganzen aus

%d. Jetzt werden die einzelnen Teile (Implementierung) beschrieben,   
%die Teile, die Routine sind, kommen i.d.R. in den Anhang,   
%diejenigen, wo Du eigene Ideen eingebracht hast, kommen in den   
%Haupttext, weil das Deine Fähigkeit zeigt, komplexe Probleme zu lösen
%)


%5. Resultate
%- Zeitmessungen/Komplexität von Algorithmen bestimmen
%- Spezifikation erfüllt: demonstrieren
%- andere Resultate


%6. Diskussion
%- Möglichkeiten und Grenzen des Ansatzes
%- Vergleich mit dem Stand der Technik
%- Was hat man mit dem Projekt gelernt, was über den Stand der   
%Technik hinausgeht


%7. Zusammenfassung und Ausblick

\documentclass{scrartcl}
\usepackage[ngerman]{babel}
\usepackage[T1]{fontenc}
\usepackage{lmodern}



\title{Brachytheraphy}
\author{Stefan Müller}
\date{August 2018}

\begin{document}

\begin{titlepage}
\maketitle
\end{titlepage}

	\begin{abstract}
	Diese Arbeit behandelt eine Simulation über die Brachytheraphy angewandt auf die Leber. Hierbei soll überprüft werden, ob die Arbeit von Johanna Stratemeier ``...`` über.. im Einklang mit den Punkten ist, die Ärzte auswählen würden. Hierzu werden Ärzte hinzugenommen, die mit diesem Eingriff bekannt sind und ihn auch an dieser 			Simulation durchführen können. Desweiteren soll es hier um den bisherigen Stand der Technik bezüglich der Simulatoren gehen. Weiterhin werden Materialien und Methoden der Simulation präsentiert, die auch auf die Auswahl der Game Engine eingehen. Zuletzt werden ihnen Resultate der Simulation präsentiert, diskutiert und es wird einen 			Ausblick auf eventuelle spätere Erweiterungen der Simulation gegeben.
	\end{abstract}
	
	\begin{page}
	
	\end{page}